\documentclass{article}
\usepackage{graphicx}
\usepackage{siunitx}
\graphicspath{ {images/} }
\title{Geosynchronous Sattelites}
\author{Grant Curell}
\begin{document}
\maketitle{}
\section{Problem}
What if you want to examine a satellite that simply stays stationary over the same place on the Earth at all times? In other words, a satellite whose period is the same as the Earth’s 24-hour period? Can you do it? Such satellites do exist. They’re very popular for communications, because they’re always orbiting in the same spot relative to the Earth; they don’t disappear over the horizon and then reappear later. They also allow for the satellite-based global positioning system, or GPS, to work. In cases of stationary satellites, the period, T, is 24 hours, or about 86,400 seconds. Can you find the radius a stationary satellite needs to have?
\\\\
Holzner, Steven. Physics I For Dummies (For Dummies (Math \& Science)) (p. 134). Wiley. Kindle Edition.
\\\\
\section{Solution}
\[ T=86,400s \]
\[ F=\frac{Gm_1m_2}{r^2} \]
\[ G=6.67*10^{-11}N*m^2/kg^2 \]
\[ M_{Earth}=5.98*10^24kg \]
\[ a_c=\frac{v^2}{r} \]
\[ \frac{m_1v^2}{r}=\frac{Gm_1m_2}{r^2} \]
\[ v^2=\frac{Gm_2}{r} \]
\[ v=\frac{2\pi*r}{T} \]
\[ (\frac{2\pi*r}{T})^2=\frac{Gm_2}{r} \]
\[ \frac{4\pi^2r^3}{T^2}=Gm_2 \]
\[ \frac{4\pi^2r^3}{T^2}=Gm_2 \]
\[ \frac{4\pi^2r^3}{86400^2}=6.67*10^{-11}*5.98*10^24 \]
\[ r=4.23*10^7 \]
\end{document}
